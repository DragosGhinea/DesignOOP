\chapter{Introduction}

\section{Motivation}

I have a strong interest in object-oriented programming, particularly in the realm of design patterns, which make complex software easier to understand and maintain. I firmly believe that one of the most effective ways to deepen understanding of a subject is by teaching it to others. Thus, for quite some time, I've been eager to develop educational resources focusing on design patterns. As a programmer, I'm also enthusiastic about creating practical and functional solutions. Using the knowledge I've gained throughout my college studies, I aim to build a full-stack application utilizing various technologies.

\section{Purpose}

The purpose of the application is to provide a platform for learning and creating courses on design patterns. Initially, the application served as a repository of courses with content hardcoded into the system. However, to facilitate easier editing and management of course content, I opted to separate the content from the design. As a result, courses are now dynamically generated from JSON data.
\\\\
\noindent To support this approach, custom tools were developed, including an integrated editor with custom linting and a graphics creator. These tools enhance the user experience and streamline the course creation process.
\\\\
\noindent The application aims to offer both official courses for learning design patterns and tools for creating new courses. These tools can be used by a diverse audience, including experienced developers seeking to create educational materials and beginners looking to create their own resources. By embracing the "learning by teaching" \cite{learn-by-teaching} methodology, users can easily create, share, and receive feedback on their course materials.
\\\\
\noindent Furthermore, by separating concerns, the web platform is responsible for managing the design, ensuring a consistent look and feel across all courses. This approach eases the burden on course creators, allowing them to focus solely on content creation.

\section{Similar Platforms}

The platforms listed below offer functionalities similar to those provided by the application described in this study.

\begin{itemize}
    \item \href{https://refactoring.guru/design-patterns}{Refactoring Guru} \\
    Refactoring Guru offers comprehensive learning resources on design patterns, featuring intuitive illustrations and UML diagrams. While it provides valuable code examples and clear explanations, its resources are limited as well as the possibility to contribute. In contrast, this platform aims to create a dynamic environment where materials can be easily created and modified while maintaining consistency.
    
    \item \href{https://www.oodesign.com/}{OODesign} \\
    OODesign is a static web application that offers information on design patterns and principles. Although its content may be relevant, the application's design lacks dynamism and interactivity.
    
    \item \href{https://www.drawio.com/}{draw.io} \\
    While not directly related to design patterns, the interactive graphics tool created in my platform is inspired by it. A useful tool for creating diagrams, but it lacks the customization options required for this platform. By creating my own graphics tool, I can create dynamic components that users can interact with, rather than relying on static image diagrams.
\end{itemize}

\noindent It's important to note that the platform described in this thesis aims to complement rather than replace the websites mentioned above. It serves as both a resource for deepening knowledge and a user-friendly starting point.