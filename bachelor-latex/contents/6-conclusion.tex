\chapter{Conclusions and Future Work}

\section{Conclusions}

In this thesis, I have documented the creation of a full-stack web application. The final product is a platform for visualizing design patterns through courses, which can be read and created within the application. The platform includes official courses and tools for creating more educational content. I've detailed the entire process, from choosing technologies to overcoming development challenges. This platform is useful for both students and teachers, offering a way to learn and teach design patterns. It separates course content from design, allowing course creators to focus on content while the platform handles the design. The source code is available on GitHub under the MIT license \cite{designoop}.
\\\\
\noindent Building a full stack application like this has been a goal of mine since I started programming. Creating a fully functional application from start to finish is a skill I believe I should have as a graduating computer science student. Through this project, I gained significant knowledge and experience, learning to use and integrate various technologies. I also focused on ensuring the application's security, performance, and ease of maintenance and extension.
\\\\
\noindent Although this was a complex and demanding project for one person, it lays the groundwork for an open-source community to contribute. This collaboration can improve the project, making it easier to maintain and manage with the shared expertise of experienced individuals.

\newpage
\section{Future Work}

The web platform leaves room for improvement and future work. If the project becomes popular within the open-source community, there will be many opportunities to improve the application. Here are some ideas for expanding the platform:

\begin{itemize}
\item \textbf{User Activity Tracking}: Currently, user accounts are only used for administrative purposes. This can be expanded to track user activity, such as course progress and completion.

\item \textbf{Course Recommendations}: The platform can be expanded to include course recommendations, allowing users to discover new courses based on their interests. This feature can be implemented using a recommendation engine that analyzes the previously mentioned user activity to recommend courses.

\item \textbf{Course Ratings and Reviews}: The platform can be expanded to include course ratings and reviews, allowing users to interact with the courses. Giving feedback can help improve the quality of the courses in the future.

\item \textbf{Quizzes}: Currently, the platform concentrates on design patterns courses, but it can be expanded to include quizzes as well. They would be a great addition to the platform and have a seamless integration with the courses. After a course or a group of courses is finished, the user can take a quiz to test their knowledge.

\item \textbf{Roadmaps}: When enough courses are established, branching in different directions and covering different topics, roadmaps can be created. These roadmaps can guide users through the courses, helping them understand the order in which they should take the courses.

\item \textbf{Community Forum}: Currently, the platform encourages user interaction externally, on either GitHub or Discord. A forum integrated within the platform could be a great addition. If a standalone forum is deemed unnecessary, the web platform could at least benefit from a section that unifies the discussions from external applications, such as a page that allows you to search through both GitHub issues and Discord topics at the same time.

\item \textbf{More Themes}: The application currently has a light and a dark theme. More themes can be added to the platform, allowing users to choose the one that suits them best. Besides the global themes for the application, course-specific themes may be added, providing themes that change only the course design while keeping the same content (e.g., a theme that makes the course more compact, one that adds more white space to it).

\item \textbf{Offline Mode}: The visited courses can be cached and made available offline. The course editor could be fully functional offline as well. Measures for this have already been taken, as the linting and suggestions are done on the client side.

\item \textbf{User Experience}: The user experience can be improved by adding more animations and transitions. The design of the courses themselves is also constantly evolving based on feedback received from users. The course and graphic editors can also be extended and modified to provide a better user experience for content creators. More intricate and specific features will be added to the editors as the platform grows and users request them.

\item \textbf{Design Patterns Editor}: Extending on the idea of helping users understand design patterns and interact with them more often, a special editor that highlights design patterns in the code can be developed. This editor could also offer suggestions on how to improve the code by applying design patterns. This idea can be implemented either as an editor inside the web platform or even as an extension to existing code editors.
\end{itemize}

