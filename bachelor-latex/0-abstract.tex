\begin{abstractpage}

\begin{abstract}{romanian}
Design pattern-urile reprezintă un aspect fundamental al dezvoltării software, oferind soluții și strategii testate pentru problemele comune întâlnite în proiectarea și în implementarea aplicațiilor. Ele nu numai că facilitează dezvoltarea unor sisteme mai eficiente și mai ușor de întreținut, dar și promovează standardele și practicile recomandate în industrie. Platforma web prezentată în acest studiu este dedicată redactării și explorării de cursuri despre design pattern-uri în contextul programării orientate pe obiecte. Furnizând un ecosistem integrat de instrumente, inclusiv un editor de cursuri și un generator de grafice interactive, aceasta facilitează procesul de elaborare a conținutului educațional. Un aspect crucial al platformei este abordarea sa în privința design-ului, conceput pentru a fi consistent și abstractizat, garantând astfel uniformitatea aspectului vizual al cursurilor și oferind o experiență de învățare fluidă și accesibilă pentru utilizatori. Această lucrare detaliază întregul proces de dezvoltare al unei aplicații, acoperind aspecte esențiale legate de frontend, backend, baze de date, precum și aspecte de securitate, pentru a oferi o imagine completă a construcției și a funcționalităților platformei.
\end{abstract}

\begin{abstract}{english}
Design patterns represent a fundamental aspect of software development, offering tested solutions and strategies for common problems encountered in designing and implementing applications. Not only do they facilitate the development of more efficient and easily maintainable systems, but they also promote industry-recommended standards and practices. The web platform presented in this study is dedicated to creating and exploring courses on design patterns in the context of object-oriented programming. By providing an integrated ecosystem of tools, including a course editor and an interactive graphics generator, it simplifies the process of developing educational content. A crucial feature of the platform is its approach to design, conceived to be consistent and abstracted, thus ensuring the uniformity of the visual elements of courses and providing a seamless and accessible learning experience for users. This thesis details the complete process of developing an application, covering essential aspects related to the frontend, backend, databases, as well as security considerations, to provide a comprehensive picture of the construction and functionalities of the platform.
\end{abstract}

\end{abstractpage}